\begin{frame}{Contribution 4: Learning Curve Analysis}
    \begin{block}{Objective}
        Evaluate how much data is required to achieve a calibration-free BP estimation algorithms.
    \end{block}
\end{frame}

\begin{frame}{Contribution 4: Learning Curve Analysis}{Methodology}
    \begin{figure}
        \includesvg[width=0.8\columnwidth]{lcanalysis/lc analysis split.svg}
    \end{figure}
    \begin{itemize}
        \item MLP Model
        \item Heartbeat Input
    \end{itemize}
\end{frame}

\begin{frame}{Contribution 4: Learning Curve Analysis}{Results}
    \begin{figure}[htbp]
        \includesvg[width=0.4\columnwidth]{lcanalysis/lc analysis SBP MAE.svg}
        \includesvg[width=0.4\columnwidth]{lcanalysis/lc analysis SBP SD.svg}

        \includesvg[width=0.4\columnwidth]{lcanalysis/lc analysis DBP MAE.svg}
        \includesvg[width=0.4\columnwidth]{lcanalysis/lc analysis DBP SD.svg}
    \end{figure}
    Improvement for SBP MAE only
\end{frame}


\begin{frame}{Contribution 4: Learning Curve Analysis}{Results}
    Power model
    \begin{equation}\label{eq:fitted curve}
        y=\num{28.533} - \num{9.518} x^{\num{0.295}}
    \end{equation}
    \begin{figure}[htbp]
        \includesvg[width=0.6\columnwidth]{lcanalysis/fitted model.svg}
    \end{figure}

    \pause
    \begin{itemize}
        \item $42 \times 21.52 \approx 904$ patients
        \item Low diversity dataset
    \end{itemize}
\end{frame}

\begin{frame}{Contribution 4}{Results}
    \begin{block}{Main Finding \#2}
        Improvements in data are more effective at improving performance for non-invasive blood pressure monitoring.
    \end{block}
\end{frame}


